\documentclass[a4paper]{article}

%%%%%%%%%%%%%%%%%%%%%%%%%%%%%%%%%%%%%%%%%%%%%%%%%%%%%%%%%%%%%%%%%%%%%%%%%%%%
% Some common includes. Add additional includes you need.
%%%%%%%%%%%%%%%%%%%%%%%%%%%%%%%%%%%%%%%%%%%%%%%%%%%%%%%%%%%%%%%%%%%%%%%%%%%%
\RequirePackage{ngerman}
\RequirePackage[utf8]{inputenc}
\RequirePackage[T1]{fontenc}
\RequirePackage[margin=23mm,bottom=30mm]{geometry}
\RequirePackage{graphicx}
\RequirePackage{amsmath,amsfonts,amssymb,amsthm}
\RequirePackage{hyperref}

%%%%%%%%%%%%%%%%%%%%%%%%%%%%%%%%%%%%%%%%%%%%%%%%%%%%%%%%%%%%%%%%%%%%%%%%%%%%
% Defines for mathematical notation. Add additional defines as needed.
%%%%%%%%%%%%%%%%%%%%%%%%%%%%%%%%%%%%%%%%%%%%%%%%%%%%%%%%%%%%%%%%%%%%%%%%%%%%
\def\O{\mathcal{O}}
\def\sort{\mathrm{sort}}
\def\scan{\mathrm{scan}}
\def\dist{\mathrm{dist}}

\setlength{\parindent}{0cm}
\renewcommand{\refname}{Quellen}
%%%%%%%%%%%%%%%%%%%%%%%%%%%%%%%%%%%%%%%%%%%%%%%%%%%%%%%%%%%%%%%%%%%%%%%%%%%%
% Definition of the assignment header
%%%%%%%%%%%%%%%%%%%%%%%%%%%%%%%%%%%%%%%%%%%%%%%%%%%%%%%%%%%%%%%%%%%%%%%%%%%%
\input{/Users/larspetersen/JWGU/Prg/header.tex}
%%%%%%%%%%%%%%%%%%%%%%%%%%%%%%%%%%%%%%%%%%%%%%%%%%%%%%%%%%%%%%%%%%%%%%%%%%%%

% Set option "german" or "english", depending on what language the
% default texts should be in.
\ExecuteOptions{german}
\ProcessOptions

% Enter the lecture name and semester
\lecture{Einf\"uhrung in die Programmierung}
\semester{Winter 2016/2017}

% Enter your data: Name, Matrikelnummer (student ID number) and group
\student{Lilian Mendoza de Sudan, Lars Petersen}{5625448, 6290157}{11}
% Tutorin: sabrinasafre@gmail.com

% Which assignment is this?
\assignment{7}

% The environment "exercise" takes one parameter (the exercise number). 
% This way you can skip exercises if you like. Example:
% 
% \assignment{3}
% \begin{exercise}{8}
% ...
% \end{exercise}
% 
% The solution to exercise 3.8 (3rd assignment, 8th exercise) goes where 
% the dots are.


\begin{document}



\begin{exercise}{1}

Dokumentation der Laufzeit in Sekunden f\"ur die Sortierung eines Arrays mit 500 Zahlen $-$ erzeugt \"uber
die Methode \texttt{generate\_random\_numbers}. Die Algorithmen wurden dabei in Ihrer schlanken Form getestet, also ohne decorator-Funktion.

\begin{center}
	\begin{tabular}{| c | c | c |}
		\hline
		Bubblesort & Insertionsort & Quicksort\\ \hline \hline
		0.12563 & 0.03512 & 0.00442\\ \hline
	\end{tabular}
\end{center}


Dokumentation der UI-Testf\"alle f\"ur die Sortierung:

\begin{center}
	\begin{tabular}{| p{2.5cm} | p{2.2cm} | p{10cm} |}
		\hline
		Funktionalit\"at & Eingabe & Verhalten des Programms\\ \hline \hline
		
		Wahl zur Eingabe der Liste & 
		& In der Konsole wird eine Benutzereingabe angezeigt, in der der Benutzer seine Wahl
		angeben kann (0 oder 1). \\ \hline
		
		Wahl zur Eingabe der Liste & $\emptyset$
		& Nutzer erh\"alt den Hinweis, dass seine Eingabe leer war und wird zu einer neuen Eingabe
		aufgefordert. Zudem wird ihm mitgeteilt, dass er durch einen KeyboardInterrupt das Programm
		beenden kann. \\ \hline
		
		Wahl zur Eingabe der Liste & Literal $\notin \{\emptyset, 0, 1\}$
		& Nutzer erh\"alt den Hinweis, dass seine Eingabe nicht konform war und wird zu einer
		neuen Eingabe aufgefordert.\\ \hline
		
		Wahl zur Eingabe der Liste & 0
		& Der Computer erzeugt ein Array mit 200 Eintr\"agen im Bereich [1 ... 20]. Dieses wird dem
		Nutzer angezeigt, und er wird zur Eingabe der gew\"unschten Algorithmen aufgefordert.\\ \hline
		
		Wahl zur Eingabe der Liste & 1
		& Der Nutzer wird zur Eingabe einer Liste zu sortierender Zahlen aufgefordert.\\ \hline
		
		Wahl zur Eingabe der Liste & Keyboard Interrupt
		& Das Programm wird mit einer Auskunft zum KeyboardInterrupt beendet.\\ \hline
		
		Eingabe der Liste & 
		& In der Konsole wird eine Benutzereingabe mit Beispiel angezeigt, in der der Benutzer
		seine zu sortierende Liste eingeben kann. \\ \hline
		
		Eingabe der Liste & $\emptyset$
		& Nutzer erh\"alt den Hinweis, dass seine Eingabe leer war und wird zu einer neuen Eingabe
		aufgefordert. Zudem wird ihm mitgeteilt, dass er durch einen KeyboardInterrupt das Programm
		beenden kann. \\ \hline
		
		Eingabe der Liste & $\notin \{\emptyset, 0, 1\}$
		& Nutzer erh\"alt den Hinweis, dass seine Eingabe nicht konform war und wird zu einer
		neuen Eingabe aufgefordert.\\ \hline
		
		Eingabe der Liste & 0
		& Der Computer erzeugt ein Array mit 200 Eintr\"agen im Bereich [1 ... 20]. Dieses wird dem
		Nutzer angezeigt, und er wird zur Eingabe der gew\"unschten Algorithmen aufgefordert.\\ \hline
		
		Eingabe der Liste & 1
		& Der Nutzer wird zur Eingabe einer Liste zu sortierender Zahlen aufgefordert.\\ \hline
		
		Eingabe der Liste & Keyboard Interrupt
		& Das Programm wird mit einer Auskunft zum KeyboardInterrupt beendet.\\ \hline

		\end{tabular}

	\begin{tabular}{| p{2.5cm} | p{2.2cm} | p{10cm} |}
		\hline
		Funktionalit\"at & Eingabe & Verhalten des Programms\\ \hline \hline
		
		Spielzug \texttt{pass} & pass
		& Mit diesem Befehl nimmt der Benutzer keine Bewegung oder \"Anderung vor.
		Die Syntax der Eingabe wird gepr\"uft. Nicht korrekte Eingaben ziehen eine neue Eingabe
		nach sich. \\ \hline
		
		Spielzug \texttt{move} & move: number, city1, city2
		& Mit diesem Befehl nimmt der Benutzer keine Bewegung oder \"Anderung vor.
		Die Syntax und die Semantik der Eingabe (St\"adte existieren, St\"adte sind verbunden,
		Anzahl Manager in city1 ist gr\"o\ss{}er gleich number) wird gepr\"uft. Nicht korrekte Eingaben
		ziehen eine neue Eingabe nach sich. \\ \hline
		
		Spielzug \texttt{build} & build: city
		& Die Syntax und die Semantik der Eingabe (St\"ad existiert, St\"adt hat nicht schon ein
		Hotel) wird gepr\"uft. Nicht korrekte Eingaben
		ziehen eine neue Eingabe nach sich. \\ \hline
		
		Spielzug \texttt{hire} & hire: city
		& Die Syntax und die Semantik der Eingabe (St\"ad existiert, Anzahl der Spieltage reicht noch
		aus, um Zug auszuf\"uhren) wird gepr\"uft. Nicht korrekte Eingaben
		ziehen eine neue Eingabe nach sich. Eine automatische Berechnung erfolgt in den
		\emph{gesperrten} Tagen. \\ \hline
		
		Berechnung des Tagesgewinns &
		& Am Ende eines Spieltages wird der Tagesgewinn berechnet und angezeigt. \\ \hline
		
		Anzeige des Spielstatus am Ende eines Tages &
		& Am Ende eines Tages wird eine Statistik mit der Anzahl der Manager, Hotels, den
		den potenziellen und dem aktuellen Gewinn je Stadt angezeigt. \\ \hline
		
		Pr\"ufung der Syntax f\"ur Befehle &
		& F\"ur jeden definierten Befehl wird die Syntax der Eingabe gepr\"uft und abgeglichen.
		Nicht konforme Angaben werden nicht akzeptiert. Dies schlie\ss{}t insbesondere Sonderzeichen
		ein. \\ \hline
		
		Aufruf der Hilfe & help!
		& Jederzeit kann der Spieler eine Hilfe anzeigen lassen. \\ \hline
	
		Beginn eines neuen Spiels & new game!
		& Hiermit kann der Spieler jederzeit eine neue Runde beginnen. \\ \hline
		
		Spielabbruch auf Wunsch des Spielers & quit!
		& Hiermit kann der Spieler jederzeit das Spiel beenden. \\ \hline
		
		Berechnung des Gesamtgewinns &
		& Am Ende eines Spieltages wird der bis dahin erzielte Gewinn ausgegeben. \\ \hline
		
		Pr\"ufung auf Highscore &
		& Am Ende des Spiels wird zusammen mit dem Namen des Spielers gepr\"uft,
		ob die erreichte Punktzahl unter den 10 besten bisherigen Ergebnissen ist. \"Uber
		das Ergebnis wird der Spieler informiert. \\ \hline

		Darstellung der Highscores &
		& Am Ende des Spiels wird die Highscore-Liste angezeigt.\\ \hline
				
		Abspeichern der Highscore-Liste &
		& Am Ende des Spiels wird der Name des Spielers abgefragt, welcher dann mit der
		erreichten Punktzahl in die High-Score-Datein eingef\"ugt wird, falls die erreichte
		Punktzahl hoch genug ist. \\ \hline
		
		Regul\"ares Spielende &
		& Nachdem die Highscore Liste dargestellt wurde, verabschiedet sich das Spiel vom Spieler.
		\\ \hline
		
	\end{tabular}
\end{center}

\end{exercise}

%\begin{thebibliography}{99}
%\bibitem{Test}
%Hallo
%\end{thebibliography}

%\begin{center}
%\includegraphics[width=6cm,angle=270]{chart.eps}
%\end{center}

\end{document}
