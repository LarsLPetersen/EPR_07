\documentclass[a4paper]{article}

%%%%%%%%%%%%%%%%%%%%%%%%%%%%%%%%%%%%%%%%%%%%%%%%%%%%%%%%%%%%%%%%%%%%%%%%%%%%
% Some common includes. Add additional includes you need.
%%%%%%%%%%%%%%%%%%%%%%%%%%%%%%%%%%%%%%%%%%%%%%%%%%%%%%%%%%%%%%%%%%%%%%%%%%%%
\RequirePackage{ngerman}
\RequirePackage[utf8]{inputenc}
\RequirePackage[T1]{fontenc}
\RequirePackage[margin=23mm,bottom=30mm]{geometry}
\RequirePackage{graphicx}
\RequirePackage{amsmath,amsfonts,amssymb,amsthm}
\RequirePackage{hyperref}

%%%%%%%%%%%%%%%%%%%%%%%%%%%%%%%%%%%%%%%%%%%%%%%%%%%%%%%%%%%%%%%%%%%%%%%%%%%%
% Defines for mathematical notation. Add additional defines as needed.
%%%%%%%%%%%%%%%%%%%%%%%%%%%%%%%%%%%%%%%%%%%%%%%%%%%%%%%%%%%%%%%%%%%%%%%%%%%%
\def\O{\mathcal{O}}
\def\sort{\mathrm{sort}}
\def\scan{\mathrm{scan}}
\def\dist{\mathrm{dist}}

\setlength{\parindent}{0cm}
\renewcommand{\refname}{Quellen}
%%%%%%%%%%%%%%%%%%%%%%%%%%%%%%%%%%%%%%%%%%%%%%%%%%%%%%%%%%%%%%%%%%%%%%%%%%%%
% Definition of the assignment header
%%%%%%%%%%%%%%%%%%%%%%%%%%%%%%%%%%%%%%%%%%%%%%%%%%%%%%%%%%%%%%%%%%%%%%%%%%%%
\input{/Users/larspetersen/JWGU/Prg/header.tex}
%%%%%%%%%%%%%%%%%%%%%%%%%%%%%%%%%%%%%%%%%%%%%%%%%%%%%%%%%%%%%%%%%%%%%%%%%%%%

% Set option "german" or "english", depending on what language the
% default texts should be in.
\ExecuteOptions{german}
\ProcessOptions

% Enter the lecture name and semester
\lecture{Einf\"uhrung in die Programmierung}
\semester{Winter 2016/2017}

% Enter your data: Name, Matrikelnummer (student ID number) and group
\student{Lilian Mendoza de Sudan, Lars Petersen}{5625448, 6290157}{11}
% Tutorin: sabrinasafre@gmail.com

% Which assignment is this?
\assignment{7}

% The environment "exercise" takes one parameter (the exercise number). 
% This way you can skip exercises if you like. Example:
% 
% \assignment{3}
% \begin{exercise}{8}
% ...
% \end{exercise}
% 
% The solution to exercise 3.8 (3rd assignment, 8th exercise) goes where 
% the dots are.


\begin{document}



\begin{exercise}{1}

Dokumentation der UI-Testf\"alle f\"ur die Sortierung:

\begin{center}	
	\begin{tabular}{| p{2.5cm} | p{3.2cm} | p{9cm} |}
		\hline
		Funktionalit\"at & Eingabe & Verhalten des Programms\\ \hline \hline
		
		\textbf{Wahl zur Eingabe der Liste} & 
		& In der Konsole wird eine Benutzereingabe angezeigt, in der der Benutzer seine Wahl
		angeben kann (0 oder 1). \\ \hline
		
		Wahl zur Eingabe der Liste & $\emptyset$
		& Nutzer erh\"alt Hinweis, dass er etwas eingeben muss. Er wird zu einer neuen Eingabe
		aufgefordert. Hinweis: Programm wird durch KeyboardInterrupt das Programm beendet. \\ \hline
		
		Wahl zur Eingabe der Liste & Literal enth\"alt Element $\notin \{0, 1\}$ \newline Bsp. \texttt{a}
		& Nutzer erh\"alt den Hinweis, dass seine Eingabe nicht konform war und wird zu einer
		neuen Eingabe aufgefordert.\\ \hline
		
		Wahl zur Eingabe der Liste & 0
		& Array mit 200 Eintr\"agen im Bereich [1 ... 20] wird erzeugt. Dieses wird angezeigt.
		Nutzer wird zur Eingabe der Algorithmen aufgefordert.\\ \hline
		
		Wahl zur Eingabe der Liste & 1
		& Der Nutzer wird zur Eingabe einer Liste zu sortierender Zahlen aufgefordert.\\ \hline
		
		Wahl zur Eingabe der Liste & Keyboard Interrupt
		& Das Programm wird mit einer Auskunft zum KeyboardInterrupt beendet.\\ \hline
		
		\textbf{Eingabe der zu sortierenden Liste} & 
		& In der Konsole wird eine Benutzereingabe mit Beispiel angezeigt, in der der Benutzer
		seine zu sortierende Liste eingeben kann. \\ \hline
		
		Eingabe der zu sortierenden Liste & $\emptyset$
		& Nutzer erh\"alt den Hinweis, dass seine Eingabe leer war und wird zu einer neuen Eingabe
		aufgefordert. Zudem wird ihm mitgeteilt, dass er durch einen KeyboardInterrupt das Programm
		beenden kann. \\ \hline
		
		Eingabe der zu sortierenden Liste & Literal enth\"alt Elemente die nicht in float umgewandelt
		werden k\"onnen
		& Nutzer erh\"alt den Hinweis, dass seine Eingabe nicht konform war und wird zu einer
		neuen Eingabe aufgefordert (ValueError).\\ \hline
		
		Eingabe der zu sortierenden Liste & \texttt{-1\_+1\_999.0}
		& Die Eingabe wird in ein Array umgewandelt. Der Nutzer wird danach zur Eingabe der
		anzuwendenden Algorithmen aufgefordert.\\ \hline
		
		Eingabe der zu sortierenden Liste & \texttt{4\_4\_4\_4}
		& Eingabe wird in ein Array umgewandelt. Der Nutzer wird zur Angabe der
		Algorithmen aufgefordert. Arrays mit mehrfach gleichen Eintr\"agen werden sortiert. \\ \hline
		
		Eingabe der zu sortierenden Liste & \texttt{4}
		& Eingabe wird in ein Array umgewandelt. Der Nutzer wird danach zur Angabe der
		Algorithmen aufgefordert. Arrays mit nur einem Eintrag werden sortiert. \\ \hline
		
		Eingabe der zu sortierenden Liste & \texttt{-1\_+1\_999.0}
		& Die Eingabe wird in ein Array umgewandelt. Der Nutzer wird danach zur Angabe der
		Algorithmen aufgefordert.\\ \hline
		
		Eingabe der zu sortierenden Liste & \texttt{0\_\_1}
		& Aufgrund der \"ubersch\"ussigen Leerzeichen erh\"alt der Nutzer den Hinweis, dass
		seine Eingabe nicht konform war (ValueError). Er wird zu einer neuen Eingabe
		aufgefordert.\\ \hline
		
		Eingabe der zu sortierenden Liste & Keyboard Interrupt
		& Das Programm wird mit einer Auskunft zum KeyboardInterrupt beendet.\\ \hline
	\end{tabular}
	
	
	\begin{tabular}{| p{2.5cm} | p{3.2cm} | p{9cm} |}
		\hline
		\textbf{Eingabe der anzuwendenden Algorithmen} &
		& In der Konsole wird eine Benutzereingabe mit Beispiel angezeigt, in der der Benutzer
		seine zu Auswahl an anzuwendenden Algorithmen eingeben kann.\\ \hline
		
		Eingabe der anzuwendenden Algorithmen & $\emptyset$ &
		Nutzer erh\"alt den Hinweis, dass seine Eingabe leer war und wird zu einer neuen Eingabe
		aufgefordert. Zudem wird ihm mitgeteilt, dass er durch einen KeyboardInterrupt das Programm
		beenden kann. \\ \hline
		
		Eingabe der anzuwendenden Algorithmen & \texttt{0\_1\_2\_3} &
		Die Eingabe ist korrekt. Der Nutzer wird aufgefordert, mit <Enter> die einzelnen Algorithmen
		nacheinander ablaufen zu lassen.\\ \hline
		
		Eingabe der anzuwendenden Algorithmen & \texttt{0\_\_1} &
		Aufgrund der \"ubersch\"ussigen Leerzeichen erh\"alt der Nutzer den Hinweis, dass
		seine Eingabe nicht konform war. Er wird zu einer neuen Eingabe aufgefordert.\\ \hline
		
		Eingabe der anzuwendenden Algorithmen & Literal enth\"alt Element $\notin \{0, 1, 2, 3\}$
		\newline Bsp. \texttt{0\_;} &
		Nutzer erh\"alt den Hinweis, dass seine Eingabe leer war und wird zu einer neuen Eingabe
		aufgefordert. \\ \hline
		
		Eingabe der anzuwendenden Algorithmen & \texttt{0\_0} &
		Die Eingabe ist korrekt (Mehrfachnennung ist m\"oglich, f\"uhrt aber nicht dazu, dass der
		entsprechende Algorithmus mehrfach ausgef\"uhrt wird). Der Nutzer wird aufgefordert, mit
		<Enter> die einzelnen Algorithmen nacheinander ablaufen zu lassen. Bubblesort wird nur
		einmal ausgef\"uhrt. \\ \hline
		
		Eingabe der anzuwendenden Algorithmen & Keyboard Interrupt
		& Das Programm wird mit einer Auskunft zum KeyboardInterrupt beendet.\\ \hline
		
		\textbf{Anzeige der Ergebnisse} & korrekte Eingabe bei Algorithmenauswahl
		& Die jeweiligen Algorithmen werden durchgef\"uhrt und deren Ergebnisse in der
		Konsole ausgegeben. \\ \hline
		
		Anzeige der Ergebnisse & korrekte Eingabe bei Algorithmenauswahl: \newline \texttt{0}
		& Bubblesort wird durchgef\"uhrt. Das eingegebene Array wird angezeigt. Das sortierte
		Array wird ebenfalls angezeigt. Die Laufzeit wird ausgegeben. Das sortierte Ergebnis von
		\texttt{np.sort} wird angezeigt. Es gibt einen Hinweis, ob die beiden sortierten Arrays
		\"ubereinstimmen. \\ \hline
		
		Anzeige der Ergebnisse & korrekte Eingabe bei Algorithmenauswahl: \newline \texttt{1}
		& Insertionsort wird durchgef\"uhrt. Das eingegebene Array wird angezeigt. Das sortierte
		Array wird ebenfalls angezeigt. Die Laufzeit wird ausgegeben. Das sortierte Ergebnis von
		\texttt{np.sort} wird angezeigt. Es gibt einen Hinweis, ob die beiden sortierten Arrays
		\"ubereinstimmen. \\ \hline
		
		Anzeige der Ergebnisse & korrekte Eingabe bei Algorithmenauswahl: \newline \texttt{2}
		& Quicksort wird durchgef\"uhrt. Das eingegebene Array wird angezeigt. Das sortierte
		Array wird ebenfalls angezeigt. Die Laufzeit wird ausgegeben. Das sortierte Ergebnis von
		\texttt{np.sort} wird angezeigt. Es gibt einen Hinweis, ob die beiden sortierten Arrays
		\"ubereinstimmen. \\ \hline
		
		Anzeige der Ergebnisse & korrekte Eingabe bei Algorithmenauswahl: \newline \texttt{3}
		& Bubblesort wird f\"ur zehn zuf\"allig gew\"ahlten Zahlen durchgef\"uhrt. Das eingegebene
		Array wird in der Konsole angezeigt. Das sortierte Array wird ebenfalls dort angezeigt. Die
		Laufzeit wird ausgegeben. Das sortierte Ergebnis von \texttt{np.sort} wird angezeigt.
		Es gibt einen Hinweis, ob die beiden sortierten Arrays \"ubereinstimmen.
		Das Sortierverfahren wird zudem durch wandernde Balken in einem Diagramm
		visualisiert. \\ \hline	
		\end{tabular}
	\end{center}

\newpage

	Screenshot zum Autruf von \texttt{tests.py} mit \texttt{numpy.random.seed(0)} zum Testen
	der Sortieralgorithmen:

	\begin{center}
		\includegraphics[scale=.8]{Test_Screenshot_01.png}
	\end{center}

\newpage
	
	Screenshot zum Autruf von \texttt{tests.py} zum Testen der Sortieralgorithmen:

	\begin{center}
		\includegraphics[scale=.8]{Test_Screenshot_02.png}
	\end{center}

\vspace*{4ex}
\hrule
\vspace*{.5ex}
\hrule
\vspace*{2ex}

Dokumentation der Laufzeit in Sekunden f\"ur die Sortierung eines Arrays mit 500 Zahlen $-$ erzeugt \"uber
die Methode \texttt{generate\_random\_numbers}.

\begin{center}
	\begin{tabular}{| c | c | c |}
		\hline
		Bubblesort & Insertionsort & Quicksort\\ \hline \hline
		0.12563 & 0.03512 & 0.00442\\ \hline
	\end{tabular}
\end{center}



\end{exercise}

\end{document}
